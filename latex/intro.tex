\documentclass[main]{subfiles}

\begin{document}
    {\Huge \textcolor{teal}{Introduction}\\}

    \textbf{Jacobi Method}, also know as \textit{Jacobi's iteration method}, is an iterative numerical 
    method that can be employed to solve linear algebraic systems of equations that are strictly \textbf{diagonally dominant}.
    Consider the system of equations:
    \\        

%------------------------------------------------------------------------------    

    \begin{align*}\label{eq1}
        &a_{11}x_1 + a_{12}x_2 + \cdots + a_{1n}x_n = b_1 \quad\quad\quad\quad\quad\quad\quad\quad(1.1)\\
        &a_{21}x_1 + a_{22}x_2 + \cdots + a_{2n}x_n = b_2  \quad\quad\quad\quad\quad\quad\quad\quad(1.2)\\
        &\;\;\vdots \hspace{15mm} \vdots \hspace{10mm} \ddots \hspace{7mm} \vdots \hspace{10mm} \vdots\quad\quad\quad\quad\quad\quad\quad\quad\quad\quad\vdots \\
        &a_{n1}x_1 + a_{n2}x_2 + \cdots + a_{nn}x_n = b_n \quad\quad\quad\quad\quad\quad\quad\quad(1.n)
    \end{align*}


%------------------------------------------------------------------------------
    \vspace{19mm}  

    The above system can be represented as:

    \begin{equation}
        A\textbf{x} = \textbf{b}
    \end{equation}
%------------------------------------------------------------------------------
    \vspace{10mm}
    
    where, 

    \begin{equation*}
        A = \begin{bmatrix}
            a_{11} & a_{12} & \cdots & a_{1n} \\
            a_{21} & a_{22} & \cdots & a_{2n} \\
            \vdots & \vdots & \ddots& \vdots \\
            a_{n1} & a_{n2} & \cdots & a_{nn}
        \end{bmatrix},
        \quad
        \textbf{x} = \begin{bmatrix}
            x_{1}  \\
            x_{2}  \\
            \vdots \\
            x_{n} 
        \end{bmatrix},
        \quad
        \textbf{b} = \begin{bmatrix}
            b_{1}  \\
            b_{2}  \\
            \vdots \\
            b_{n} 
        \end{bmatrix}
    \end{equation*}

    \clearpage

%------------------------------------------------------------------------------
    Let A be further written as a sum of 3 matrices $L$, $U$, $D$ as ``$A = L + D + U$``, 
    \\
    where $L, D, U$ are strictly lower triangular, diagonal and upper triangular matrices, respectively:
    \\

    $\quad\quad A \quad\quad\quad\quad\quad = \quad\quad\quad\quad\; L \quad\quad\quad\quad +  \quad\quad\quad\quad D \quad\quad\quad\quad\quad + \quad\quad\quad\quad U$ 
    \begin{equation*}
        \begin{bmatrix}
            a_{11} & a_{12} & \cdots & a_{1n} \\
            a_{21} & a_{22} & \cdots & a_{2n} \\
            \vdots & \vdots & \ddots& \vdots \\
            a_{n1} & a_{n2} & \cdots & a_{nn}
        \end{bmatrix} =
        \;
        \begin{bmatrix}
            0      & 0      & \cdots & 0 \\
            a_{21} & 0      & \cdots & 0 \\
            \vdots & \vdots & \ddots& \vdots \\
            a_{n1} & a_{n2} & \cdots & 0
        \end{bmatrix}+
        \;
        \begin{bmatrix}
            a_{11} & 0      & \cdots & 0 \\
            0      & a_{22} & \cdots & 0 \\
            \vdots & \vdots & \ddots & \vdots \\
            0      & 0      & \cdots & a_{nn}
        \end{bmatrix}+
        \;
        \begin{bmatrix}
            0      & a_{12} & \cdots & a_{1n} \\
            0      & 0      & \cdots & a_{2n} \\
            \vdots & \vdots & \ddots & \vdots \\
            0      & 0      & \cdots & 0
        \end{bmatrix}
    \end{equation*}


%------------------------------------------------------------------------------
    \vspace{14mm}

    Then, from \textit{Jacobi Method}, the $i^{th}$ approximation, $\textbf{x}^{(i)}$, is computed as:

    \begin{equation}
        \textbf{x}^{(i)} = D^{-1}(\textbf{b} - (L+U) \textbf{x}^{(i-1)})
    \end{equation}

%------------------------------------------------------------------------------
\vspace{15mm}
Consider the following system of equations: 
% \\ with diagonally :

    \begin{align}\label{eq1}
        a_{11}x_1 + a_{12}x_2 + a_{13}x_3 &= b_1 \\
        a_{21}x_1 + a_{22}x_2 + a_{23}x_3 &= b_2 \\
        a_{31}x_1 + a_{32}x_2 + a_{33}x_3 &= b_3 
    \end{align}

%------------------------------------------------------------------------------
\vspace{13mm}
Let $a_{11}, a_{22}, a_{33}$ be the numerically dominant coefficients for the above system.
Then on solving for $x_1, x_2, x_3$, they can be written as:

    \begin{align}\label{eq1}
        x_1  &= \frac{1}{a_{11}}(b_1 - a_{12}x_2 - a_{13}x_3) \\
        x_2  &= \frac{1}{a_{22}}(b_2 - a_{21}x_1 - a_{23}x_3)  \\
        x_3  &= \frac{1}{a_{33}}(b_3 - a_{31}x_1 - a_{32}x_2) 
    \end{align}

Let us start with some initial approximation $\textbf{x}^{(0)}$ 

    \begin{align*}\label{eq1}
    \textbf{x}^{(0)} = \begin{bmatrix}
        x_1^{(0)}\\
        x_2^{(0)}\\
        x_3^{(0)}
    \end{bmatrix}     
    \end{align*}

    Then the first approximation, $\textbf{x}^{(1)}$, can be calculated by substituting $x_1^{(0)}, x_2^{(0)}, x_3^{(0)}$ in RHS of (6), (7) and (8):

    \begin{align*}
        x_1^{(1)}  &= \frac{1}{a_{11}}(b_1 - a_{12}x_2^{(0)} - a_{13}x_3^{(0)}) \\
        x_2^{(1)}  &= \frac{1}{a_{22}}(b_2 - a_{21}x_1^{(0)} - a_{23}x_3^{(0)})  \\
        x_3^{(1)}  &= \frac{1}{a_{33}}(b_3 - a_{31}x_1^{(0)} - a_{32}x_2^{(0)}) 
    \end{align*}


    Once we obtain $\textbf{x}^{(1)}$, we can repeat the above process to compute $\textbf{x}^{(2)}$, 
    \\$\textbf{x}^{(3)}$, \ldots, $\textbf{x}^{(i)}$, \ldots using :

    \begin{align*}
        x_1^{(i)}  &= \frac{1}{a_{11}}(b_1 - a_{12}x_2^{(i-1)} - a_{13}x_3^{(i-1)}) \\
        x_2^{(i)}  &= \frac{1}{a_{22}}(b_2 - a_{21}x_1^{(i-1)} - a_{23}x_3^{(i-1)})  \\
        x_3^{(i)}  &= \frac{1}{a_{33}}(b_3 - a_{31}x_1^{(i-1)} - a_{32}x_2^{(i-1)}) 
    \end{align*}

    Which takes the following Matrix form,
    \\

    \begin{equation*}
    \textbf{x}^{(i)} = \begin{bmatrix}
         \frac{1}{a_{11}} &     0            &      0           \\
            0             & \frac{1}{a_{22}} &      0            \\
            0             &     0            & \frac{1}{a_{33}}       
    \end{bmatrix} \left(
    \begin{bmatrix}
        b_{1}  \\
        b_{2}  \\
        b_{3} 
    \end{bmatrix} - 
    \begin{bmatrix}
            0      &     a_{12} &      a_{13}           \\
            a_{21} &     0      &      a_{23}            \\
            a_{31} &     a_{32} &       0       
    \end{bmatrix}
    \begin{bmatrix}
        x_{1}^{(i-1)}  \\
        x_{2}^{(i-1)}  \\
        x_{3}^{(i-1)} 
    \end{bmatrix} \right)    
    \end{equation*}

%------------------------------------------------------------------------------
    \vspace{2mm}
    i.e.

    \[
        \textbf{x}^{(i)} = D^{-1}(\textbf{b} - (L+U) \textbf{x}^{(i-1)})
    \]

%------------------------------------------------------------------------------
\clearpage

\end{document}