\documentclass[main]{subfiles}

\begin{document}
    
    \noindent{\Huge \textcolor{teal}{Discussions}} \\

    There are a few points worth mentioning about the \textit{Jacobi Method} algorithm and the program 
    \textcolor{cyan}{\textit{Solve\_Lin\_Alg\_Jacobi\_method.m}}:
    \\
    \begin{itemize}
        
        \item There can various criteria for \textit{convergence} for a numerical method. 
        For most matrix based iterative algorithms the standard convergence criterion is that the spectral radius of the iteration 
        matrix is less than 1:\\\\
        For \textit{Jacobi Method} it is,   
        $\rho \left( D^{-1} \left( L + U \right) \right) < 1$
        \\\\
        Line: {\small $9$} in the code performs this check. It will ensure convergence criterion is met! 


        \item It is important to shed light upon the fact that \textit{diagonal dominance} is just a 
        \textbf{sufficient} condition and not a \textbf{necessary} condition for convergence. The \textbf{necessary} criterion 
        has been stated in the previous bullet point.

        \item This algorithm is based on similar grounds as the popular \textcolor{cyan}{FIXED POINT ITERATION METHOD} 
        also known as \textit{Iteration Method}.

        \item The program is only allowed to have a maximum of $500$ iterations (line: $25$ and $27$) therefore the program might terminate without 
        outputting the solution to desired accuracy. This is done to ensure that the program terminates and does not fall in an infinite loop!
    \\

    \end{itemize}
    
    \noindent{\Huge \textcolor{teal}{Conclusion}} \\\\
    Jacobi method has been described in quite enough detail for a general understanding of the method. Apart from that, a working program has been shown that performs soft computations 
    based on this algorithm. \\\\
    It is also worth noting that Jacobi method is a good starting point for comprehending other iterative methods to solve a system of equations but it is not the finest 
        of algorithms that can solve a large system efficiently. There are better methods like \textcolor{teal}{\textit{Gauss-Seidel}} (nearly twice as fast as \textit{Jacobi}) and \textcolor{teal}{\textit{Successive-Over Relaxation}} (even faster than \textit{Gauss-Seidel}!). 
    

\end{document}